\documentclass{article}
\usepackage[utf8]{inputenc}
\usepackage{cite}

\title{Top Down Operator Precedence}

\author{Vaughan R. Pratt \thanks{Work reported herein was supported in part at
    Stanford by the National Science Foundation under grant no GJ 992, and the
    Office of Naval Research under grant number N-OOO14-67-A0112-0057 NR
    044-402; by IBM under a post-doctoral fellowship at Stanford; by the IBM
    T.J.  Watson Research Center, Yorktown Heights, N.Y,; and by Project MAC, an
    MIT research Program sponsored by the Advanced Research Projects Agency,
    Department of Defense, under Office of Naval Research Contract Number
    NOO014-70-0362-OO06 and the National Science Foundation under contract
    number GJOO-4327. Reproduction in whole or in part is permitted for any
    purpose of the United States Government.}}

\date{Massachusetts Institute of Technology 1973}
\bibliographystyle{apalike}

\begin{document}
\maketitle
\section{Survey of the Problem Domain}

There is little agreement on the extent to which syntax should be a
consideration in the design and implementation of programming languages. At one
extreme, it is considered vital, and one may go to any lengths
\cite{10.5555/1098667, mckeeman1970} to provide adequate syntactic
capabilities. The other extreme is the spartan denial of a need for a rich
syntax \cite{10.1145/321574.321575}. In between, we find some language
implementers willing to incorporate as much syntax as possible provided they do
not have to work hard at it \cite{10.1007/BF00264291}.

\par

In this paper we present what should be a satisfactory compromise for a
respectably large proportion of language designers and implementers. We have in
mind particularly

\begin{enumerate}
  \item those who want to write translators and interpreters (soft, firm or
    hardwired) for new or extant languages without having to acquire a large
    system to reduce the labor, and
  \item those who need a convenient yet efficient language extension mechanism
    accessible to the language user.
\end{enumerate}

\par

The approach described below is very simple to understand, trivial to implement,
easy to use, extremely efficient in practice if not in theory, yet flexible
enough to meet most reasonable syntactic needs of users in both categories (1)
and (2) above. (What is "reasonable" is addressed in more detail
below). Moreover, it deals nicely with error detection.

\par

\bibliography{top_down_operator_precedence.bib}
\end{document}
